\chapter{Literature Review}
\label{chp:LIT}
Under this section, we analyse the various studies which have been performed in relation to our research question.  The section has been split into various ``themes'' pertaining to our research question. The ``themes'' include, but are not limited to: studies on Mother-to-child transmission risk factors, studies on the interventions applied to the mothers only, the discourse on the interventions applied to infants only, discourses on a combined intervention approach to both mothers and infants, and the evolution of HBV vertical transmission modeling, with vaccination and treatment, over time.

%\subsection{Vertical Transmission Risk Factors}

\subsection{HBV Mother-to-Child Transmission Risk Factors}
According to  \cite{Pan2012}, an efficient way to prevent vertical transmission of hepatitis B is to profile the risk factors of this mode of transmission, and further, to identify the mothers who possess the most risk so as to administer the interventions to prevent HBV MTCT. Eventually, they listed: maternal level of HBV DNA $>200,000$ IU/mL, positive test(s) for the HB envelope Antigen(HBeAg) and HB surface Antigen(HBsAg), pregnancy complications such as threatened pretem labour, or prolonged labour, and failure of the immunoprophylaxis in children who had received it, as the risk factors to consider. Other authors such as \cite{wen2013mother} have agreed with \cite{Pan2012} in terms of maternal DNA levels being one of the most important risk factors. 

%Concerns have been raised as to whether to consider breastfeeding as a risk factor for perinatal transmission. Whilst some authors say there is no such risk, for instance in \cite{beasley1975evidence}, others think it is worth further investigation 


%Others, for instance, [ref] are skeptical about factors such as breastfeeding being a risk since there exist little evidence to make such strong assertions.

\subsection{Infant Vaccination}
Several authors for instance, those of \cite{tran2009management,andersson2015mother, xu2013nextstep,shimakawa2016mother}, have posited that it will be prudent to concentrate on preventing mother-to-child transmission of hepatitis B if eradication is to be achieved in the near future. On the said matter, the authors of \cite{Pan2012} conducted a systematic review of articles published between $1975-2011$ on HBV mother-to-child transmission. They deduced that by administering Hb immunoglobulin alongside the Hb vaccine between the time of birth and at most $12$ hours after exposure, combined with the routine vaccination regimen between $6-12$ months, to an infant, will provide an approximate $95\%$ chance of preventing the perinatal transmission of HBV from their HBsAg-positive pregnant mother. Other studies have also shown that mother-to-child transmission is not totally preventable, even though, by combining strategies such as antiviral therapy to highly viremic mothers, alongside a full vaccination strategy of a birth vaccine and the routine vaccine administered to infants can provide better protection to the infants\cite{tran2009management} in preventing the infection. Authors of the paper \cite{tran2009management} provided a more enlightening HBV mother-to-child transmission strategy thus: for babies at high risk of perinatal infection from HBeAg infected mothers, a birth dose alongside the routine vaccine may provide enough protection. However, for babies born to HBsAg positive mothers, HB immunoglobulin should be included in the birth regimen. Furthermore, it has been suggested that the second dose must be administered in time, as a study by \cite{tharmaphornpilas2009increasedRisk} has proven that a delay in receiving the subsequent vaccines might cause a reduction in the protection of the infants against HBV transmission. 

These interventions under study in our research question in Section \ref{section: research questions} are therefore worth our attention since research spanning over at least the past $2-3$ decades seem to be corroborating the same assertion. 

\subsection{Treatment of Infected Mothers}
Several drugs have been under use to suppress HB viral replication. Telbuvindine, Lamivudine and Tenofivir are some of such examples. Viral replication is a key factor in MTCT. Concerning the use of tenofovir as immunoprophylaxis for reducing HBV MTCT, a recent study in the paper \cite{pan2016TenofivirToPrevent} has concluded that it might be a good option in the future, as predicted by the authors in \cite{xu2013nextstep}. The same primary author, in a different trial, the results of which were published in \cite{pan2012telbivudine}, revealed that telbuvindine is a safe drug for the suppression of vertical transmission in a group of pregnant women with HBeAg positive status.
\documentclass[12pt,a4paper]{article}
\usepackage{a4wide}
\usepackage[latin1]{inputenc}
\usepackage{psfrag}
\usepackage{subfigure}
\usepackage{epsfig}
\usepackage{amsmath,amssymb,amsfonts,amsthm,latexsym,fancyhdr} 
\usepackage{multicol}
\usepackage{enumerate}
%\usepackage[square]{natbib}
%\usepackage{textcomp}
\usepackage[T1]{fontenc}
\usepackage{setspace}
\usepackage[citecolor=magenta,colorlinks=true,urlcolor=blue,a4paper]{hyperref}
\special{papersize=210mm,297mm}
%
\textwidth 6.5in
\textheight 9in
\topmargin 0in
\headheight 0in
\oddsidemargin 0in
\evensidemargin 0in
\parskip 0.5\baselineskip
\parindent 0pt
\renewcommand{\familydefault}{\sfdefault}
\linespread{1.0}
\author{HJB Njagarah$^{\ddag}$\footnote{Corresponding author's email: johnhatson@sun.ac.za}\\
{\it {\small $^\ddag$Department of Mathematics, University of Stellenbosch, Private Bag X 1, Matieland, 7602, South Africa}}}
\title{The Cholera epidemic in South Africa}
\begin{document}
\maketitle
\section{Introduction}
Water-borne infections have posed a heavy budgetary burden on the Republic of South Africa and the health of individuals. A number of water-borne infections currently prevalent include Cholera, Schistosomiasis, cryptosporiosis, Shigella and other water related infections such as malaria. 


Reporting of cholera epidemic often results in international stigmatisation and sanctions related to travel in and out  of affected countries. This often results in under reporting of cholera cases in affected countries.

Cholera is an infectious diseases caused by the bacteria species \textit{Vibrio cholerae}. The main route of infections is related to sanitation and access to clean water. The infection is mainly spread by drinking contaminated water or eating food contaminated with the bacteria. Therefore,cholera can be classified as a water-borne/food borne disease. The bacteria present in the faecal matter of an infected person is the main source of infection. Once one is infected, the main site affected in the human body is the gastrointestinal tract.   

The symptoms include, acute watery diarrhoea, vomiting, suppression of urine, rapid (severe) dehydration, fall in 
blood pressure, cramps in legs and abdomen, subnormal temperatures and complete collapse. If uncontrolled through 
prompt medical intervention, death can occur within $24$ hours of onset. 

Individuals who are protected from cholera via access to clean water can still acquire the disease through other transmission routes. In this respect, in case of an epidemic vaccination in an ideal solution. The justification for this observation can be traced fro studies conducted in Haiti where clean water was distributed to a small subset of the population in one study and vaccination of an identical number of individuals in another. Vaccination was observed to produce a much bigger impact on the case counts as opposed to sole supply of clean water. The main reason is that whereas individuals who receive clean water still remain susceptible to infection, vaccinated individuals may not easily contract the pathogen and pass it on.  

The persistence and seasonality of the epidemic can be attributed to health carriers of the pathogen, climate and 
migration and movement patterns of individuals. We examine each of these factors to ascertain how they affect the 
cholera epidemic. The healthy carriers of the pathogen \textit{V. cholerae} are symptomless individuals who 
intermittently excrete the pathogen at relatively short durations of 6 to 15 days  with the maximum period being 
between $30$ to $40$ days. However, there are also chronic convalescent carriers and these have been observed to 
excrete the pathogen intermittently for periods of $4$ to $15$ months. 

The viability of \textit{V. cholerae} is surface water has been observed to vary from $1$ hour to $13$ days. Its 
survival is entirely centred around the chemical, biological and physical characteristics of the given stream of 
estuarine water. Although the viability of \textit{V. cholerae} may be short in polluted aquatic environment, 
faecal contamination from victims of the epidemics and healthy carriers of the pathogen continue to reinforce their 
population in water. 

\subsection{A survey of the cholera epidemic in  South Africa}
By around July 2001, the cholera epidemic had spread  to seven of the nine provinces of South Africa. The major 
affected areas were the North and Southern parts of KwaZulu Natal where the outbreak had occurred as early as 
August 2000. The affected area of KwaZulu Natal had 99\%  of all the 106224 case reported nationally. Currently, KwaZulu Natal, Norther province, Eastern cape, Mpumalang and Gauteng are some of the severely affected provinces with water-borne infections including cholera. Of these all KwaZulu Natal is still the mostly affected province.

The characteristic areas in the provinces that at mainly affected include townships and informal settlements where there is rapid urbanisation yet no adequate access to clean drinking water, poor hygiene, over crowding with respect to living conditions, lack of safe food preparation, handling and storage, famine and flooding.

Cholera not only affects the population with regard to morbidity, mortality and Disability adjusted life years, it also imposes serious social and economic impacts. It can cost the government of South Africa billions of rand to eradicate, working time is lost due to absenteeism  of employees who may be affected or attending to patients. The lost working time may affect production in the industry and consequently tax revenue.
\cite{Blower,Pascual,Seas}
Cholera can be prevented through proper disposal of human excreta through building and using proper sanitation systems, proper and safe preparation and handling of food. Although proper sanitation systems are vital in containing the epidemic, this alone may not be effective if no effective primary health care education is enhanced.

\section{Pathology of \textit{Vibrio cholerae}}
When  \textit{Vibrio cholerae} enters the digestive system, it embeds itself in the villi of the absorptive intestinal cells and releases cholera toxin. The cholera toxin (CT) is an enterotoxin made up of five B-subunits that form a spore that fits one A-subunit \cite{Zhang}.

physiological responses and symptoms that follow release of cholera toxin include stimulation of mucosal lining of the intestine to secrete fluids. This causes vomiting and profuse watery diarrhoea that has a ``rice water'' quality. These result in excessive dehydration which can be fatal. In general if not treated, death does occur 50-70\% of the time.
\subsection{\textit{Vibrio Cholerae}}
\textit{V. cholerae} is a ``comma'' shaped Gram-negative with a single flagellum for movement. There are several strains of \textit{V. cholerae}, some of which are pathogenic and some are non pathogenic. The most widely sweeping pathogenic strain is the \textit{V. cholerae} serotype 01 El Tor N 16961 strain that causes the pandemic disease cholera.
The latest pathogenic serotype 0139 was discovered in 1992. The El Tor strain was active in the seventh and the most recent pandemic of cholera from 1960s to 1970s as well as the early 1990s along with serotype 0139, both displaying resistance to multiple drugs.

\subsection{Cholera in South Africa} 
Some of the first cases of Cholera in South Africa were detected around 1973 in the gold mines \cite{Isaacson}. The introduction of cholera was attributed to migrant labourers from the then cholera endemic countries such as Malawi, Mozambique and Angola who  had come to work in the gold mines. 
\bibliographystyle{plain}
%\bibliographystyle{apalike}
\bibliography{Cholera}


\end{document}
\chapter*{\huge\scshape  Abstract} 
\addcontentsline{toc}{chapter}{Abstract}
%\vspace*{-0.5cm}
%\hrule
%\vspace*{0.5cm}
 \begin{center}
 {\large \textbf{Combinatorics of Oriented Graphs}}\\
 \vspace*{0.6cm}
 Isaac Owino Okoth
 
 \emph{Department of Mathematical Sciences,\\
 University of Stellenbosch,\\
 Private Bag X1, Matieland 7602, South Africa.}\\
  \vspace*{0.6cm}
 Dissertation: PhD\\
  \vspace*{0.2cm}
  December 2015 
  \end{center}
In this dissertation, a number of enumeration problems are tackled. Firstly, we enumerate labelled trees with a given indegree sequence and also by the number of sources and sinks. Du and Yin, Shin and Zeng, and Wagner proved an elegant formula for the number of labelled trees with respect to a given indegree sequence, where each edge is oriented from a vertex of  lower label towards a vertex of higher label. We prove a formula for the number of such trees with respect to a given indegree sequence, or number of sources, such that a given vertex $r$ is a sink of degree $d$. Analogous results for labelled trees with two marked vertices are also given. We find formulas for the mean and variance of the number of sinks in these trees. We obtain a differential equation and a functional equation satisfied by the generating function for these trees.  We also extend these results to noncrossing trees.

%Remmel and Williamson obtained a generating function for the number of these labelled trees in which both indegree and outdegree sequences are given, though a closed formula is still non-existent. In this thesis, we obtain some explicit formulas for the number of these labelled trees. We concentrated on some trees with only one source and it remains an open problem to determine a unifying formula for the number of trees in which both indegree and outdegree are given at the same time.

Secondly, we enumerate labelled trees by path lengths. We prove some new formulas concerning reachable vertices. Among other results, we obtain a counting formula for the number of labelled trees on $n$ vertices in which exactly $k$ vertices are reachable from a given vertex $i$ and also the average number of vertices that are reachable from a specified vertex in labelled trees of order $n$ for any large $n.$ Some known results in the enumeration of labelled trees by the number of sources and sinks also follow from our theorems as corollaries.

Thirdly, this thesis deals with enumeration of coloured Husimi graphs and cacti. We provide a proof for a formula that counts the number of connected cycle-free families of $k$ set partitions of $[n]$ satisfying a certain coherence condition and then establish a bijection between these families  and the set of labelled free $k$-ay cacti with a given vertex-degree distribution. We also show that the formula counts coloured Husimi graphs in which there are no blocks of the same colour that are incident to one another. We extend the work to coloured oriented cacti and coloured cacti.

\chapter*{\huge\scshape  Opsomming} 
\addcontentsline{toc}{chapter}{Opsomming}
%\vspace*{-0.5cm}
%\hrule
%\vspace*{0.5cm}
 \begin{center}
 {\large \textbf{Kombinatorika van Geori\"{e}nteerde Grafieke}}\\
 (\esquote{\emph{Combinatorics of Oriented Graphs}})\\
 \vspace*{0.6cm}
 Isaac Owino Okoth
 
 \emph{Departement Wiskundige Wetenskappe,\\
 Universiteit van Stellenbosch,\\
 Privaatsak X1, Matieland 7602, Suid Afrika.}\\
  \vspace*{0.6cm}
 Proefskrif: PhD\\
  \vspace*{0.2cm}
  Desember 2015 
  \end{center}



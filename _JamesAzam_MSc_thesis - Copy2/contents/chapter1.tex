\chapter{Introduction}
\label{chp:Intro}
Of all chronic viral infections in the world, hepatitis B (HBV), a liver disease, is arguably the most popular. 
In some parts of the world, particularly, in most parts of Asia and sub-Saharan Africa, hepatitis B is highly endemic. A systematic review by \cite{ott2012GlobalEpidemiology} has estimated HBV to have caused over 600,000 deaths in the past. It is a common cause for liver cancer, liver cirrhosis, and some other liver-related diseases. The WHO has consequently listed HBV as part of its major health priorities due to its high mortality and morbidity rates.

HBV is acquired  through contact with blood, serum, or bodily fluids of highly viremic individuals. This is termed horizontal transmission. It can also be transmitted from a mother to her child(vertical) either prenatally, or neonatally. The infection transmission dynamics is such that, mother-to-child(vertical) transmission is the most common\cite{tran2009management}. Some authors, for instance, \cite{tran2009management,andersson2015mother} have posited that, in order to reduce the burden of the disease, it is important to reduce the number of infections arising from the vertical transmission route. Also,  chronic hepatitis B is inversely related to the age at which individuals acquire the infection\cite{tran2009management} . It is therefore pertinent that vertical transmission is reduced so as to further deplete the number of chronically infected individuals in the long run. 

The disease is not evenly distributed across the globe, as it is almost rare in most advanced countries whilst remaining highly endemic in sub-Saharan Africa and parts of Asia\cite{medley2001hepatitis}. The World Health Organization (W.H.O) has made it a priority to reduce the prevalence of the disease to certain levels within the next decade in the highly endemic regions of the world. Several countries have recognised the health impact of the disease and are continually implementing revised strategies geared towards reducing the hepatitis B prevalence. China, for instance, has a target of reducing its HBV prevalence below 1\% within the next decade \cite{pang2010DynamicalBehaviour}. Countries in sub-Saharan Africa, however, have not yet put in effective measures to reduce the impact of the disease as this is evident in the results of studies like \cite{ott2012GlobalEpidemiology} which estimated the prevalence of the disease by world regions. 

HBV vaccination has been present since the early 1980's \cite{tharmaphornpilas2009increasedRisk}. Studies have shown there has been a noticeable reduction in the disease prevalence and this can be attributed to the introduction of childhood immunization. In many low-income countries, the vaccine is usually administered in three doses from week $6$, after birth, till week $14$. Studies have shown that, this routine vaccination strategy is only efficient if the infants remain uninfected till the $6$th week after birth. However, to infants born to chronic HB positive mothers, vaccination will only be effective if it is administered at the time of birth, or within $24$ hours after birth. Other researchers have corroborated this strategy through their studies. It has further been established that a first dose at birth, combined with the routine vaccination between weeks $6$ to $14$ without delay, will provide approximately $97\%$ efficiency in protecting the infants \cite{tharmaphornpilas2009increasedRisk,tran2009management}.

In the quest to prevent HBV mother-to-child transmission, the risk factors involved in the process must be thoroughly investigated. One of such factors is the maternal DNA which has won the highest mention in articles over the years \cite{tran2009management,Pan2012}. 

Studies like \cite{tran2009management,thio2015global} have predicted that the risk of hepatitis b vertical transmission could be reduced if more efficient methods, for example, antiviral therapy, are put in place to reduce the HBV DNA levels below the established level of $10^7$ IU/mL. A high level of maternal DNA can be inferred from most studies to mean any level above $10^7$ IU/mL or $8\log_{10}$ copies/mL. At high maternal DNA levels, it has been noticed that vertical transmission begins to increase with a positive correlation. A recent cohort trial by \cite{pan2016TenofivirToPrevent} has produced very convincing results that seem to affirm the stance that, a reduced maternal DNA level below $10^7$ IU/mL, combined with administering a birth vaccine, as well as, the routine vaccine starting at week 6 after birth, is likely to produce desirable results in the quest to reduce mother-to-child transmission as a way to eventually eradicate HBV globally. 

In conclusion,  a campaign for the eradication of HBV mother-to-child transmission has been long standing. Various prevention strategies have been proposed towards the amelioration of vertical transmission of HBV in the bid to eradicate the disease. This thesis therefore aims to provide a mathematical point of view in contribution to the ongoing discourse. For this to be achieved, this work shall answer the questions that follow:

\subsection{Research Question}
\label{section: research questions}
This study aims to answer the question and sub-questions: 
\begin{enumerate}[1.]
	\item Can Hepatitis B be eradicated from Sub-Saharan Africa (SSA) within a couple of generations, given the intervention is limited to: 
	\begin{enumerate} [(a)]
		\item treatment of the mothers only,  \label{sub question 1}
		\item administering a combination of the birth vaccine and the routine vaccine to the infants only, \label{sub question 2}
		\item a combination of sub-questions \ref{sub question 1} and \ref{sub question 2}
	\end{enumerate}
\end{enumerate}

\subsection{Aim and Objectives of the Research}
By answering the research question, this thesis will aim to serve as a mathematical backing in the prevailing campaign towards considering the revision of the hepatitis B prevention strategies in sub-Saharan Africa and other parts of the world where the disease remains a burden.  

In this thesis, our objectives include:
\begin{enumerate}
	\item to examine hepatitis B vertical transmission which captures interventions applied to both the mothers and their newborns,
	\item to investigate whether the proposed interventions affect the population dynamics of the infected infants,
	\item determine the contribution of the infected infants to the disease dynamics in the general population over time,
	\item provide recommendations towards further research aimed at eradicating the disease.
	
\end{enumerate}

%%%%%%%%%%%%%%%%%%%%%%%%%%%%%%%%%%%%%%%%%%%%%%%%%%%%%%%%%%%%%%%%%%%%%%%
\section{Background on the hepatitis B virus} 
\section{Background on the hepatitis B virus Elimination} 
